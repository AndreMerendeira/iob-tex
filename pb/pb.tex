\documentclass[twocolumn]{iob_pb}

%Params
%\def\XILINX{1}
%\def\INTEL{1}

\usepackage{color,soul}
%\usepackage{array}
\usepackage{float}
\usepackage{makecell}

\usepackage[table,xcdraw]{xcolor}
\usepackage{calc}
\graphicspath{{./figures/}}

\title{IOB-MP1/2-D}
\category{MPEG1/2 - LAYER I/II AUDIO DECODER}
\confidential{}

\usepackage{eso-pic}
\newcommand\BackgroundPic{%
\put(0,0){%
\parbox[b][\paperheight]{\paperwidth}{%
\vfill
\centering
\includegraphics[width=\paperwidth,height=\paperheight,%
keepaspectratio]{bg.pdf}%
\vfill
}}}


\begin{document}
\AddToShipoutPicture*{\BackgroundPic}

\section*{\textcolor[rgb]{0,0,0}{Overview}}

The MPEG1/2 Layer I/II Audio Decoder (IOB-MP2-D) is an audio engine IP core
powered by the IObundle proprietary IOB-RV32-X processor, which supports the
standard RISC-V instruction set architecture. This IP core is capable of
real-time decoding of 1 Elementary Stream of MPEG1/2 Layer I/II encoded audio
data.

The compressed audio stream can be input using either an I2S or parallel
interface. The parallel interface data is organized in a burst/stuffing format,
where useful data is transmitted during the burst period and don't care data is
transmitted during the stuffing period. This way, the data interfaces can use
fast clocks whose frequency is unrelated to the sample rate. Additionally, the
parallel output requires an audio sample clock for synchronization. The system,
input, output and word clocks may all be asynchronous for maximum flexibility
but synchronous operation is also supported.

The IP core uses a UART for outputting messages useful during the IP integration
phase. The UART can also be used to receive configuration and control data from
a host. Configuration and control can also be exerted via a dedicated AXI Lite
interface.

\section*{\textcolor[rgb]{0,0,0}{Features}}
\begin{itemize}
\item Compliant with the ISO/IEC 11172-3 and the ISO/IEC 13818-3 standards
\item Real time decoding of 1 stereo audio stream at 100 MHz.
\item I2S and par audio interface
\item I2S and par audio with PCM-like compressed data
\item Sample rates: 16 to 48KHz
\item Bitrates: 32 to 448/384kbps (Layer I/II)
\item 16-bit output audio resolution
\item Required memory size: 128kByte
\item Configuration, Control and Status register file with APB slave interface
\item Channel modes: Mono, joint stereo and stereo
\item Latency: 384 (Layer I) or 1152 (Layer II) audio sample periods
\item AXI Lite interface for control and configuration
\item Asynchronous or synchronous system, input, output and sample clocks
\end{itemize}

\newpage
\section*{\textcolor[rgb]{0,0,0}{Benefits}}
\begin{itemize}
\item Compact hardware implementation
\item Can fit many instances in low cost FPGAs
\item Can fit many instances in small ASICs 
\item Low power consumption
\end{itemize}

\section*{\textcolor[rgb]{0,0,0}{Block Diagram}}
\begin{figure}[!htb]
  \vspace{-.5cm}
  \centering
      {\includegraphics[width=\columnwidth]{bd.png}}
\end{figure}

% Please add the following required packages to your document preamble:
% \usepackage[table,xcdraw]{xcolor}
% If you use beamer only pass "xcolor=table" option, i.e. \documentclass[xcolor=table]{beamer}

\section*{FPGA Resources}
\ifnum\XILINX=1
\begin{table}[h]
\centering
\begin{footnotesize}
\begin{tabular}{|c|c|c|c|c|}
\hline
\rowcolor[rgb]{0,0.5,0.5}
{\bf \textcolor{white}{FPGA}} &
{\bf \textcolor{white}{LUT-6}} &
{\bf \textcolor{white}{\makecell{BRAM-18 kb}}} &
{\bf \textcolor{white}{\makecell{BRAM-36 kb}}} &
{\bf \textcolor{white}{DSPs}} \\
\hline
KU040 		& 3539  & 2 & 32 & 3  \\ \hline
\end{tabular}
\end{footnotesize}
\caption{Xilinx Kintex Ultrascale}
\end{table}
\fi

\ifnum\INTEL=1
\begin{table}[h]
\centering
\caption{Results for Intel FPGAs}
\begin{footnotesize}
\begin{tabular}{|c|c|c|c|c|}
\hline
\rowcolor[rgb]{0,0.5,0.5}
{\bf \textcolor{white}{FPGA}} &
{\bf \textcolor{white}{ALMs}} &
{\bf \textcolor{white}{M10K}} &
{\bf \textcolor{white}{M20K}} &
{\bf \textcolor{white}{DSPs}} \\
\hline
Cyclone V GT           &  &   &  &  \\ \hline
\rowcolor[HTML]{CBCEFB}
{\scriptsize Arria V}  &  &   &  &  \\ \hline
\end{tabular}
\end{footnotesize}
\end{table}
\fi

\section*{\textcolor[rgb]{0,0,0}{Deliverables}}
\begin{itemize}
\item FPGA netlist or source code (optional)
\item Example testbench
\item Implementation constraints for map, place and route
\item Demo on commercial board with Ethernet realtime MPEG 1 Layer II decoding
\item Datasheet and user documentation for system integration
\end{itemize}

\section*{\textcolor[rgb]{0,0,0}{Contact information}}
\begin{tabular}[t]{ll}
{\bf Web}:     & www.iobundle.com\\
{\bf Email}:   & info@iobundle.com\\
\end{tabular}

\vspace*{0.5cm}
\noindent
\begin{scriptsize}
Disclaimer: IObundle reserves the right to modify the current
technical specifications without notice.
\end{scriptsize}

\end{document}
